\documentclass{article}
\usepackage{amsmath}
\usepackage[margin=1in]{geometry}

\begin{document}

\title{Essence of Single Variable Calculus}
\author{Andy Chong Sam}
\date{}
\maketitle

\section{Derivatives}

\par \noindent We will first summarize derivatives. Given a function \( f(x)\), its derivative \(f'(x)\), describes the instantaneous rate of change at \(x\).
The formal definition of the derivative is as follows:

\begin{flalign}
	f'(x) = \lim_{\Delta x \to  0 } \frac{f(x+ \Delta x) - f(x)}{\Delta x}
\end{flalign}

\par\noindent From Expression (1) we can derive all the derivatives found on a textbook table:

\section{First Fundamental Theorem of Calculus}

\par \noindent \textbf{If \(f(x)\) is continuous over \([a,b]\), and \(F(x) = \int_{a}^{x} f(t) dt \), then \(F'(x)=f(x)\) over \([a,b]\).} We say that \(F\) is an antiderivative of \(f\).
\newline
\par\noindent\textbf{Derivation:}
\newline
\begin{flalign*}
	F'(x) = \lim_{\Delta x \to  0 } \frac{F(x+ \Delta x) - F(x)}{\Delta x} \\ \\
	F'(x) = \lim_{\Delta x \to  0 } \frac{\int_{a}^{x + \Delta x}f(t)dt - \int_{a}^{x}f(t)dt}{\Delta x} \\ \\
	F'(x) = \lim_{\Delta x \to  0 } \frac{\int_{x}^{x + \Delta x}f(t)dt}{\Delta x}
\end{flalign*}
\par\noindent By the \textbf{Mean Value Theorem} we know that there is a value \(c\) such that:

\begin{flalign*}
f(c) \Delta x = \int_{x}^{x + \Delta x}f(t)dt
\end{flalign*}

\par \noindent We are left with evaluating:

\begin{flalign*}
F'(x)= \lim_{\Delta x \to  0 } f(c) \Delta x
\end{flalign*}

\par \noindent By the \textbf{Squeeze Theorem} we know that as \(\Delta x\) approaches zero, \(c\) approaches \(x\), so we are left with:

\begin{flalign*}
	F'(x) = \lim_{\Delta x \to  0 } f(c) \Delta x = f(x)
\end{flalign*}

\newpage

\section {Second Fundamental Theorem of Calculus}

\par\noindent \textbf{If \(f(x)\) is continuous over \([a,b]\), then \(\int_{a}^{b} f(x) dx = F(b) - F(a)\)}.
\newline
\par\noindent \textbf{Derivation:}
\newline
\par \noindent Let \(g(x) = \int_{a}^{x}f(t)dt\). Applying the First Fundamental Theorem we know that \(g'(x) = f(x)\). If \(F(x)\) is any antiderivative of \(f(x)\) then \(F(x) = g(x) + c\). Let's evaluate \(F(b) - F(a)\):

\begin{flalign*}
	F(b) - F(a) \\
	= (g(b) + c) - (g(a) + c) \\
	= g(b) - g(a) \\
	= \int_{a}^{b} f(t) dt - \int_{a}^{a}f(t)dt \\
	=  \int_{a}^{b} f(t) dt - 0 \\
	= \int_{a}^{b} f(t) dt
\end{flalign*}
\newpage
\section{Product Rule}

\par\noindent Given \(h(x) = f(x)g(x)\), \(h'(x) = f'(x)g(x) + f(x)g'(x)\)

\begin{flalign*}
	h'(x) = \lim_{h \to  0 }\frac{f(x+h)g(x+h) - f(x)g(x)}{h}
\end{flalign*}

\par\noindent We'll add two terms that cancel each other out (but will help us derive the rule): \(-f(x+h)g(x) + f(x+h)g(x)\):

\begin{flalign*}
		h'(x) = \lim_{h \to  0 }\frac{f(x+h)g(x+h) -f(x+h)g(x) + f(x+h)g(x) - f(x)g(x)}{h} \\
		= \lim_{h \to  0 }\frac{f(x+h)(g(x+h) - g(x)) + g(x)(f(x+h)-f(x))}{h} \\
		= \lim_{h \to  0 }\frac{f(x+h)(g(x+h) - g(x))}{h} + \lim_{h \to  0 }\frac{g(x)(f(x+h))-f(x)}{h} \\
\end{flalign*}

\par\noindent From this result, we see that \(\lim_{h \to  0 } f(x+h)\) is just \(f(x)\). We can also see that \(\lim_{h \to  0 } \frac{g(x+h)-g(x)}{h}\) is \(g'(x)\).
\newline
\par\noindent We also observe that \(\lim_{h \to  0 }g(x) = g(x)\), and \(\lim_{h \to  0 }\frac{f(x+h) - f(x)}{h}\) is the definition of \(f'(x)\).

\section{Quotient Rule}

\par\noindent If \(h(x) = \frac{f(x)}{g(x)}\) then \(h'(x) = \frac{f'(x)g(x) - f(x)g'(x)}{g(x)^2}\)
\newline

\par\noindent We will apply the cross multiplication property. Given \(\frac{a}{b} - \frac{c}{d} = \frac{ad-bc}{bd}\):

\begin{flalign*}
	h'(x) = \lim_{h \to  0 } \frac{\frac{f(x+h)}{g(x+h)} - \frac{f(x)}{g(x)}}{h} \\
	=\lim_{h \to  0 } \frac{f(x+h)g(x) - f(x)g(x+h)}{g(x+h)g(x)}\;\frac{1}{h}
\end{flalign*}

\par\noindent We'll add two terms that cancel each other out (but will help us derive the rule): \(-f(x)g(x) + f(x)g(x)\):

\begin{flalign*}
	h'(x)=\lim_{h \to  0 } \frac{f(x+h)g(x) -f(x)g(x) + f(x)g(x) - f(x)g(x+h)}{g(x+h)g(x)}\;\frac{1}{h} \\
	= \lim_{h \to  0 } \frac{1}{g(x+h)g(x)}\frac{f(x+h)g(x) -f(x)g(x) + f(x)g(x) - f(x)g(x+h)}{h} \\
	= \lim_{h \to  0 } \frac{1}{g(x+h)g(x)} ( \frac{f(x+h)g(x) - f(x)g(x)}{h} + \frac{f(x)g(x) - f(x)g(x+h)}{h}) \\
	= \lim_{h \to  0 } \frac{1}{g(x+h)g(x)} ( g(x)\frac{f(x+h) - f(x)}{h} + f(x)\frac{g(x) - g(x+h)}{h})
\end{flalign*}

\par \noindent From these results we can see that: 
\newline
\par\noindent \(\lim_{h \to  0 } \frac{1}{g(x+h)g(x)}\) is \(g'(x)^2\)
\newline
\par \noindent \(\lim_{h \to  0 }(g(x)\frac{f(x+h) - f(x)}{h}\) is \(g(x)f'(x)\).
\newline
\par \noindent \(\lim_{h \to  0 }(f(x)\frac{g(x+h) - g(x)}{h}\) is \(f(x)g'(x)\).
\newpage

\section {Power Rule}

\par\noindent Given \(f(x) = x^n\), its derivative is \(f'(x) = nx^{n-1}\)
\newline
\par\noindent \textbf{Derivation:}
\begin{flalign*}
	f'(x) = \lim_{h \to  0 } \frac{(x+h)^n - x^n}{h}
\end{flalign*}
\par\noindent By the Binomial Theorem, we know that:
\begin{flalign*}
	(a+b)^n = \sum_{k=0}^{n} \frac{n!}{k!(n-k)!} a^{n-k}b^k
\end{flalign*}
\par\noindent Expanding the numerator we have:
\begin{flalign*}
	f'(x)= \lim_{h \to  0 } \frac{(x^n + nx^{n-1}h + \frac{n(n-1)}{2!}x^{n-2}h^2 + ... + nxh^{n-1} + h^n) - x^n}{h} \\
	= \lim_{h \to  0 } (nx^{n-1} + \frac{(n)(n-1)}{2!}x^{n-2}h + ... + nxh^{n-2} + h^{n-1})
\end{flalign*}
\par\noindent With the exception of \(n(x)^{n-1}\), the limit as \(h\) approaches zero of all other components is zero, for instance:


\begin{flalign*}
	\lim_{h \to  0 } (\frac{(n)(n-1)}{2!}x^{n-2}h) = 0 \;\;\;\;\;
	\lim_{h \to  0 } (nxh^{n-2}h) = 0 \;\;\;\;\;
	\lim_{h \to  0 } (h^{n-1}) = 0 \\
\end{flalign*}

\par \noindent Since \(n(x)^{n-1}\) does not depend on \(h\), we have: 

\begin{flalign*}
 \lim_{h \to  0 }(n(x)^{n-1}) = n(x)^{n-1} \\
\end{flalign*}
\par\noindent So we are left with:
\begin{flalign*}
	f'(x) = n(x)^{n-1}
\end{flalign*}

\par\noindent The First Fundamental Theorem of calculus predicts the existence of a function \(F(x)\) such that \(F'(x) = f(x)\).

\begin{flalign*}
	\frac{d}{dx}[\frac{x^{n+1}}{n+1}] = \frac{n+1}{n+1} x ^ {n+1-1} = x^n
\end{flalign*}

\par\noindent So we have: 

\begin{flalign*}
	\int x^n dx = \frac{x^{n+1}}{n+1} + C
\end{flalign*}
\newpage
\section{Sine}

\par \noindent If \(f(x) = \sin x\), then \(f'(x) = \cos x\)

\begin{flalign*}
	f'(x) = \lim_{h \to  0 } \frac{\sin(x+h) - \sin x}{h} \\
\end{flalign*}

\par \noindent Using the trig identity \(\sin (x+h) = \sin x \cos h + \cos x \sin h\), we can rewrite the numerator:

\begin{flalign*}
	f'(x) = \lim_{h \to  0 } \frac{\sin x \cos h + \cos x \sin h - \sin x}{h} \\
	= \lim_{h \to  0 }\frac{\sin x(\cos h-1) + \cos x\sin h}{h} \\
	= \lim_{h \to  0 }\frac{\sin x(\cos h - 1)}{h} + \lim_{h \to  0 }\frac{\cos x\sin h}{h} \\
	= \sin x \lim_{h \to  0 }\frac{\cos h - 1}{h} + \cos x\lim_{h \to  0 } \frac{ \sin h}{h} \\
	= -\sin x \lim_{h \to  0 }\frac{1- \cos h}{h} + \cos x\lim_{h \to  0 } \frac{ \sin h}{h} \\ 
\end{flalign*}

\par \noindent We will apply two known special limits here, namely that \( \lim_{h \to  0 } \frac{1- \cos h}{h} = 0\) and \(\lim_{h \to  0 }\frac{\sin h}{h} = 1\). So we are left with:
\begin{flalign*}
	f'(x) = \cos x
\end{flalign*}

\par\noindent The First Fundamental Theorem of calculus predicts the existence of a function \(F(x)\) such that \(F'(x) = f(x)\).

\begin{flalign*}
	\frac{d}{dx}\; [-\cos x] = - (- \sin x) = \sin x \\ \text{Note: derivative of cosine is discussed in the next section.} 
\end{flalign*}

\par\noindent So therefore:

\begin{flalign*}
	\int \sin x dx = -\cos x + C
\end{flalign*}


\section{Cosine}

\par\noindent If \(f(x) = \cos x\) then \(f'(x) = - \sin x \)

\begin{flalign*}
	f'(x) = \lim_{h \to  0 } \frac{\cos(x+h)- \cos x}{h}
\end{flalign*}

\par \noindent We can use the following trig identity, \(\cos (x+h) = \cos x \cos h - \sin x \sin h\), to rewrite the numerator:

\begin{flalign*}
	f'(x) = \lim_{h \to  0 } \frac{(\cos x \cos h - \sin x \sin h)- \cos x}{h} \\
	= \lim_{h \to  0 } \frac{\cos x(\cos h -1) - \sin x \sin h}{h} \\
	= \lim_{h \to  0 } \frac{\cos x(\cos h - 1)}{h} - \lim_{h \to  0 }\frac{\sin x \sin h}{h}	\\
	= \cos x \lim_{h \to  0 } \frac{(\cos h - 1)}{h} - \sin x \lim_{h \to  0 }\frac{\sin h}{h} \\
	= -\cos x \lim_{h \to  0 } \frac{1 - \cos h}{h } - \sin x \lim_{h \to  0 }\frac{\sin h}{h} \\
\end{flalign*}

\par \noindent We will apply two known special limits here, namely that \( \lim_{h \to  0 } \frac{1- \cos h}{h} = 0\) and \(\lim_{h \to  0 }\frac{\sin h}{h} = 1\). So we are left with:

\begin{flalign*}
	f'(x) = -\sin x
\end{flalign*}

\section{Tangent}

\par\noindent If \(f(x) = \tan x\), then \(f'(x) = \sec ^2 x\)

\begin{flalign*}
	f'(x) = \lim_{h \to  0 }\frac{\tan(x+h)-\tan x)}{h}
\end{flalign*}

\par\noindent If we apply the trig identity, \( \tan(x+h) = \frac{\tan x + \tan h}{1- \tan x \tan h} \), we get:

\begin{flalign*}
	f'(x) = \lim_{h \to  0 } \frac{\frac{\tan x + \tan h - \tan x(1 - \tan x \tan h)}{1- \tan x \tan h}}{h} \\
	= \lim_{h \to  0 } \frac{\tan h + \tan^2 x \tan h}{h(1- \tan x \tan h)} \\
	= \lim_{h \to  0 } \frac{\tan h(1 + \tan^2 x)}{h(1 - \tan x \tan h)} \\
	= \lim_{h \to  0 }(\frac{\tan h}{h}) \lim_{h \to  0 }(\frac{1+\tan^2 x}{1- \tan x \tan h}) \\
	= \lim_{h \to  0 }( \frac{\sin h}{h \cos h}) \lim_{h \to  0 }(\frac{1+\tan^2 x}{1- \tan x \tan h}) \\
	= \lim_{h \to  0 } \frac{\sin h}{h} \lim_{h \to  0 }\frac{1}{\cos h} \lim_{h \to  0 }(\frac{1+\tan^2 x}{1- \tan x \tan h}) \\
	= 1 + \tan^2 x
\end{flalign*}

\par\noindent We can simplify the result further:

\begin{flalign*}
	f'(x) = 1 + \tan^2 x =  \frac{\cos^2 x}{\cos^2 x} + \frac{\sin^2 x}{\cos^2 x} = \frac{\cos^2 x + sin^2 x}{cos^2 x}
\end{flalign*}

\par \noindent Since \(\cos^2 x + \sin^2 x = 1\), we are left with \(\frac{1}{cos^2 x}\) or \(sec^2 x\).

\begin{flalign*}
	f'(x) = \sec^2 x
\end{flalign*}

\par \noindent There is a function \(F(x)\) such that \(F'(x) = f(x)\). Let's differentiate \(\ln (\sec x)\): 

\begin{flalign*}
\frac{d}{dx} [ \ln(\sec x)]
= (\frac{1}{\frac{1}{\cos x}}) (-\frac{1}{\cos^2 x}) (-\sin x) \\
= \frac{(\cos x) (-\sin x)}{-\cos^2 x} = \frac{\sin x}{\cos x} \\
= \tan x
\end{flalign*}

\par\noindent We can therefore say that:

\begin{flalign*}
	\int \tan x dx = \ln (\sec x) + C
\end{flalign*}

\section{Natural Log}

\par\noindent If \(f(x) =  \ln x \), then \(f'(x) = \frac{1}{x}\)

\begin{flalign*}
	f'(x) = \lim_{h \to  0 }\frac{\ln (x + h) - \ln x}{h}
\end{flalign*}

\par \noindent We can apply the following rule of natural logs: \( \ln (x+h) - \ln x = \ln ( \frac{x+h}{x})\):

\begin{flalign*}
f'(x) = \lim_{h \to  0 } \frac{1}{h} \ln(\frac{x+h}{h})	
\end{flalign*}

\par \noindent Applying the power rule of natural logs we get:

\begin{flalign*}
	f'(x) =\lim_{h \to  0 } \ln(\;\;(\frac{x+h}{x})^{\frac{1}{h}}\;\;)
\end{flalign*}

\par \noindent Let \(u=\frac{h}{x}\), so \(h=ux\):

\begin{flalign*}
	f'(x) = \lim_{u \to  0 } \ln(\;\;(1 + u) ^{\frac{1}{ux}}\;\;) \\
	= \lim_{u \to  0 } \ln(\;\;(1 + u) ^{\frac{1}{u}\frac{1}{x}}\;\;)
\end{flalign*}

\par \noindent We apply the power rule one more time:

\begin{flalign*}
	f'(x) = \lim_{u \to  0 } \frac{1}{x} \ln(\;\;(1 + u)^{\frac{1}{u}}\;\;) \\
	= \frac{1}{x} \lim_{u \to  0 }\ln(1 + u)^{\frac{1}{u}}
\end{flalign*}

\par \noindent The expression \( \lim_{u \to  0 }(1+u)^{\frac{1}{u}}\) is the definition of Euler's number, \(e\), so we are left with:

\begin{flalign*}
	f'(x) = \frac{1}{x} \ln e\\
	f'(x) = \frac{1}{x}
\end{flalign*}

\par \noindent Let's take the derivative of \( \ln x\). We know that \(y= \ln x\) is basically stating that \(e^y = x\), to which we'll apply implicit differentiation:

\begin{flalign*}
	\frac{d}{dx} [e^y] = \frac{d}{dx} [x] \\
	e^y \frac{dy}{dx} = 1 \\
	\frac{dy}{dx} = \frac{1}{e^y}\\
\text{(If we plug in \(y= \ln x\))}\;\;\;\;	\frac{dy}{dx} = \frac{1}{e^{\ln x}} = \frac{1}{x} \;\;
\end{flalign*}

\par \noindent So we have:

\begin{flalign*}
	\int \ln x dx = \frac{1}{x} + C
\end{flalign*}

\section e

\par \noindent If \(f(x) = e^x\), then \(f'(x) = e^x\).

\begin{flalign*}
	f'(x) = \lim_{h \to  0 }\frac{e^{x+h} - e^x}{h} \\
	= \lim_{h \to  0 }\frac{e^{x}e^{h} - e^x}{h} 
\end{flalign*}
 
 \par \noindent Since \(e^x\) does not depend on \(h\) we can take it out of the limit evaluation:
 
 \begin{flalign*}
 	f'(x)= e^x \lim_{h \to  0 }\frac{e^{h} - 1}{h}
 \end{flalign*}

\par \noindent Let \(y=e^h - 1\), or \(e^h = y + 1\) we observe that \(\lim_{h \to  0 } y = 0\). Taking the natural log of boths sides:

\begin{flalign*}
	\ln (e^h) = \ln(y+1) \\
	h = \ln (y+1)
\end{flalign*}

\par \noindent We can substitue the above results into \(h\) and \(e^h - 1\):

\begin{flalign*}
	f'(x) = e^x \lim_{h \to  0 }\frac{e^h - 1}{h} = e^x \lim_{y \to 0} \frac{y}{\ln (y+1)} \\
	= e^x\lim_{y \to 0}\frac{1}{\frac{1}{y} \ln (y+1)} \\
	= e^x\lim_{y \to 0}\frac{1}{\ln(\;\;(y+1)^{\frac{1}{y}}\;\;)}
\end{flalign*}

\par \noindent The expression \(\lim_{y \to 0} (y+1)^{\frac{1}{y}}\) is the definition of Euler's number, so we are left with:

\begin{flalign*}
	f'(x) = e^x \frac{1}{\ln e} \\
	f'(x) = e^x
\end{flalign*}

\par\noindent It can also be said that:

\begin{flalign*}
	\int e^x dx = e^x + C
\end{flalign*}

\end{document}