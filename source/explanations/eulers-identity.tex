\documentclass{article}
\usepackage{amsmath}
\usepackage[margin=1in]{geometry}
\usepackage{amsfonts}
\usepackage{hyperref}
\usepackage{graphicx}

\begin{document}


\title{Euler's Identity}
\author{Andy Chong Sam}
\date{2021-12-10}
\maketitle

\section {Euler's Identity}

\par\noindent Euler's Identity is an equality that brings together several mathematical ideas: Euler's constant (e), the imaginary number unit (i) and the ratio of the circumference to the diameter of the circle (\( \pi \)). Expression 1 below is Euler's formula, and expression 2 is Euler's identity (where \(x = \pi \)):

 \begin{flalign}
e^{ix} = \cos x +  i \sin x
\end{flalign} 

 \begin{flalign}
e^{i\pi} + 1 = 0
\end{flalign} 

\par\noindent Deriving the identity requires the application of the Maclaurian series for cos(x), sin(x), and \(e^{x}\). The Maclaurian series is a specific case of the Taylor Series, centered at 0. The purpose of both series is to provide approximations for functions by using the n-order derivatives of the original function. The bigger the n, the closer the approximation.  Given a function f(x) the form of the Maclaurian series is as follows: 

 \begin{flalign}
f(x) =  \sum_{n=0}^{\infty} \frac{f^{(n)} (a)}{n!} (x-a)^{n}  
\end{flalign} 
 \begin{flalign*}
f(x) =  f(0) + \frac{f'(0)}{1!}x +  \frac{f''(0)}{2!}x^{2} + \frac{f'''(0)}{3!}x^{3} + \frac{f''''(0)}{4!}x^{4} + ...
\end{flalign*} 

\par\noindent The analysis of \( f^{(n)} \) reveals a cyclical pattern, taking exactly 4 differentiations to revert to the original function. 

 \begin{flalign*}
\frac{d}{dx} [\sin x] = \cos x \; \;   \;   \;       \frac{d}{dx}  [\cos x] = -\sin x \; \;   \;   \;     \frac{d}{dx}  [-\sin x] = -\cos x \; \;   \;   \;     \frac{d}{dx}  [-\cos x] = \sin x 
\end{flalign*} 

 \begin{flalign*}
\frac{d}{dx} [\cos x] = -\sin x \; \;   \;   \;       \frac{d}{dx}  [-\sin x] = -\cos x \; \;   \;   \;     \frac{d}{dx}  [-\cos x] = \sin x \; \;   \;   \;     \frac{d}{dx}  [\sin x] = \cos x 
\end{flalign*} 

\par\noindent Since the Maclaurian series is centered at 0, we can compute f(0) for all three of the functions: \(e^{0} = 1 \) , \( \cos(0) = 1 \), \( sin(0) = 1 \). We can now plug these values into expression 7 for each of the functions:

 \begin{flalign*}
\cos x = 1 - \frac{x^2}{2!} + \frac{x^4}{4!} - \frac{x^6}{6!} + \frac{x^8}{8!} ...
\end{flalign*} 

 \begin{flalign*}
\sin x = 1 - \frac{x^3}{3!} + \frac{x^5}{5!} - \frac{x^7}{7!} + \frac{x^9}{9!} ...
\end{flalign*} 

\par\noindent In expression 7, the portions of the sequence related to odd powers of n disappear, as they require the evaluation of +/- sin 0, which results in 0. On  expression 8, the portion related to even numbers disappear for the same reason. 
\newpage

\par\noindent If the Maclaurian series is applied to \( e^{ix} \), expression 10 becomes: 

 \begin{flalign*}
e^{ix} = 1 + \frac{ix}{1!} +   \frac{(ix)^2}{2!} +  \frac{(ix)^3}{3!} + \frac{(ix)^4}{4!}  +  \frac{(ix)^5}{5!} ...
\end{flalign*} 

 \begin{flalign*}
e^{ix} = 1 + \frac{ix}{1!} +   \frac{i^{2}x^2}{2!} +  \frac{i^{3}x^3}{3!} + \frac{i^{4}x^4}{4!}  + \frac{i^{5}x^5}{5!}  ...
\end{flalign*} 

\par\noindent When we raise the imaginary unit to increasing powers, a cyclical pattern is observed as well. Starting by raising i to the 1st power, the pattern resets and starts anew on the 5th power: 

 \begin{align*}
&i^{1} = i &&\\&i^{2} = -1&& \\ &i^{3} = i^{1} i^{2} = -i  \\&i^{4} = i^{2}i^{2} = 1 \\&i^{5} = i^{4}i^{1}=i \\&i^{6} = i^{2}i^{4} = -1 
\end{align*} 

\par\noindent With this in mind, we can simplify expression 10:

\begin{flalign*}
e^{ix} = 1 + ix -   \frac{x^2}{2!} -  \frac{ix^3}{3!} + \frac{x^4}{4!}  + \frac{ix^5}{5!} - \frac{x^6}{6!}  ...
\end{flalign*} 

\begin{flalign*}
e^{ix} = (1 -   \frac{x^2}{2!} + \frac{x^4}{4!} - \frac{x^6}{6!} + ... ) + i(x - \frac{x^3}{3!} + \frac{x^5}{5!} ...)
\end{flalign*} 

\par\noindent It can be observed that the pattern grouped on the left corresponds to expression 11, the Maclaurian series expansion for cos x, and the right group corresponds to expression 12, the expansion of sin x. 

\begin{flalign*}
e^{ix} =  \cos x + i\sin x
\end{flalign*} 

\par\noindent When x = \(x = \pi \) , cos x becomes -1, isin x becomes 0, leaving us with Euler's identity:

 \begin{flalign*}
e^{i\pi} + 1 = 0
\end{flalign*} 
\newpage

\end{document}


