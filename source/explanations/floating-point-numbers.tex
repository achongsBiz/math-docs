
\documentclass[12pt]{article}
\thispagestyle{empty}
\usepackage{amsmath}
\usepackage[margin=1in]{geometry}
\usepackage{amsfonts}
\usepackage{hyperref}
\usepackage{graphicx}
\usepackage{siunitx}
\usepackage{cancel}
\usepackage{xfrac}
\usepackage{listings}

\begin{document}
	\begin{center}
	\par\noindent \large \textbf{Floating Point Numbers}  [Andy Chong Sam]
\end{center}

\par\noindent The IEEE 754 standard states that floating point numbers are stored with \(32\) bits: 
\par\noindent \textbf{1 bit for the sign, 8 bits for the exponent and 23 bits for the mantissa.} 
\newline

\par\noindent As an example let's see how 320.625 is represented in binary.
\newline
\newline
\begin{minipage}[t]{.5\linewidth}
	\par\noindent \textbf{(1) Translate to Binary}
	\newline
	 \par\noindent We first represent the whole part of the number in binary so for \(320\) that is \(101000000\). 
	\newline
	\par\noindent To represent the fractional part in binary, we'll continuously multiply the fractional part by \(2\) until we get a result of 1:
	\begin{flalign*}
		0.625 \times 2 = 1.25 \text{ save the 1} \\
		0.25 \times 2 = 0.5 \text{ save the 0} \\
		0.5 \times 2 = 1.0 \text{ save the 1} \\
		\text{The last step produces a 1, so we stop.} \\
		\text{0.625 is 0.101 in binary.}
	\end{flalign*}
	\par\noindent We now have a representation for \(320.625\) in binary:
	\begin{flalign*}
	101000000.101
	\end{flalign*} 
	\par\noindent \textbf{(2) Write in Scientific Notation}
		\newline
	\par\noindent The result from part 1 becomes: 
	\begin{flalign*}
	1.01000000101 \times 10^8
	\end{flalign*}

\end{minipage}
\hspace{0.45cm}
\begin{minipage}[t]{.5\linewidth} 
	\par\noindent \textbf{(3) Add Bias} 
	\par\noindent127 is added to the exponent. The result from part 2 becomes: 
	\begin{flalign*}
		1.01000000101 \times 10^{135}
	\end{flalign*}
	\par\noindent \textbf{(4) Determine the Sign}
	\par\noindent 320.65 is positive so the sign bit will be \(0\).
	\newline
	\par\noindent \textbf{(5) Translate Exponent to Binary}
	\par\noindent From part 3, 135 in binary is:
	\begin{flalign*}
		10000111
	\end{flalign*}
	\par\noindent \textbf{(6) Extract the Mantissa}
	\par\noindent The mantissa is everything to the right of the decimal point from part 3:
	 	\begin{flalign*}
	 	01000000101
	 \end{flalign*}
 	\par\noindent \textbf{(7) Collect Result} 
 	\par\noindent Combine the results from steps 4, 5, and 6. Pad zero's to the mantissa as needed. The binary representation of 320.65 is:
 \begin{flalign*}
 \text{0 10000111 01000000101000000000000}
 \end{flalign*}
\end{minipage}
\newline
\newline
\newline
\par\noindent We can always take a binary floating point number and determine its decimal equivalent. The formula used is:
\begin{flalign*}
	d = (-1)^s (1+m)\;2^e
\end{flalign*}
\par\noindent In this expression, \(s\) is the sign bit, \(m\) is the decimal representation of the mantissa, and \(e\) is the decimal representation of the exponent. With our problem, \(s = 0\). We now translate the mantissa, \(01000000101000000000000\), back into decimal:  \(m= 2^{-2} + 2^{-9} + 2^{-11}\). For the exponent, \(10000111\) is the binary representation of 135, if we subtract the bias, \(127\), we get \(e=8\)
\newline
\par\noindent So the decimal is: \((-1)^0(\;\;1+(2^{-2} + 2^{-9} + 2^{-11})\;\;)2^8 = 320.625\)
\newpage
\par\noindent An additional example is provided here:
\newline
\newline
\framebox{
	\parbox{\linewidth}{
	\par\noindent Represent \(-23.15625\) in binary using the IEEE 754 standard.
	\newline
	\par\noindent 23 in binary is \(10111\). Let's figure out the fractional part:
	\begin{flalign*}
		0.15625 \times 2 = 0.3125 \;\;\text{save the 0} \\
		0.3125 \times 2 = 0.625 \;\;\text{save the 0} \\
		0.625 \times 2 = 1.25 \;\;\text{save the 1} \\
		0.25 \times 2 = 0.50 \;\;\text{save the 0} \\
		0.50 \times 2 = 1.0 \;\;\text{save the 1} \\
		\text{We have a 1, so we stop}
	\end{flalign*}
	\par\noindent So \(23.15625\) is \(10111.00101\).
	\newline
	\par\noindent Rewrite in scientific notation: \(10111.00101\) becomes \(1.011100101 \times 10^4\)
	\newline
	\par\noindent Add 127 bias: \(1.011100101 \times 10^4\) becomes \(1.011100101 \times 10^{131}\)
	\newline
	\par\noindent The exponent, 131, in binary is \(10000011\).
	\newline
	\par\noindent We're dealing with a negative number so the sign bit will be \(1\).
	\newline
	\par\noindent The mantissa is \(011100101\).
	\newline
	\par\noindent We know the sign bit, exponent, and  mantissa, so the binary representation will be:
	\begin{flalign*}
		11000001101110010100000000000000
	\end{flalign*}

}}
\newline
\newline
\newline
\par\noindent As with any number system some fractional amounts cannot be perfectly represented. This is not unusual even using day to day decimal math. Consider the fraction \(\frac{1}{3}\). There is no way to represent this perfectly in decimal, we'd simply get an infinite string of threes after the decimal point. 
\newline
\par\noindent Try and convert the decimal \(0.1\) into a fractional binary decimal in the manner we discussed, you will get an unending sequence of bits. In this case some form of rounding system is typically used.
\end{document}