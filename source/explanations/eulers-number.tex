\documentclass{article}
\usepackage{amsmath}
\usepackage[margin=1in]{geometry}
\usepackage{amsfonts}
\usepackage{hyperref}
\usepackage{graphicx}

\begin{document}


\title{Compounding Interest and Euler's Number (e) }
\author{Andy Chong Sam}
\date{2021-05-17}
\maketitle

\section {Compounding Interest}

\par\noindent Describing compounding interest for a rate of return r across time periods t for a fixed payment a takes the form: 

\begin{flalign}
a (1+r)^t 
\end{flalign}

\par\noindent We can derive expression 1 as follows:

\begin{flalign*}
t=0 \rightarrow a(1+r)^0=a
\end{flalign*}

\par\noindent At t=1, we need to add the original investment plus any interest earned (ar) during the period.

\begin{flalign*}
t=1 \rightarrow a + ar
\end{flalign*}

\par\noindent Now when t=2, we add the amount we started with(a+ar), to the interest earned in this meriod (a+ar)r.

\begin{flalign*}
t=2 \rightarrow (a + ar) + (a + ar)r \\ = (a + ar) (1 +r) \\= a (1 + r )(1 + r)\\=a(1+r)^2
\end{flalign*}

\par\noindent Now for t=3, the initial amount is (a + ar)(1+ r), or the amount at t=2. The interest earned at t=3 is therefore (a+ar)(1+r)r. 

\begin{flalign*}
t=3 \rightarrow (a+ar)(1+r) + (a+ar)(1+r)r \\ 
= (a+ar)(1+r)(1 +r)  \\
= a(1+r)(1+r)(1+r) \\=a(1+r)^3
\end{flalign*}

\par\noindent We can see how that the total value of an investment can be described as by: \(V=a(1+r)^{t}\), where t is the number of elapsed periods, r is the interest, and a is the initial amount.

\newpage

\section {Deriving Euler's Number (e)}

\par\noindent  Suppose that a \$1 investment makes 100 \% interest. If interest is credited once at the end of the year, then the value at the end of the year is just 1 + 1 = 2. Now suppose that the interest will be credited twice a year, the end value will be 2.25.

\begin{flalign*}
1(1+\frac{1}{2})^2 = 2.25
\end{flalign*}

\par As you increase the number of times the interest is credited, we will eventually converge at e:

\begin{flalign*}
( 1+\frac{1}{2} )^2 = 2.25 \\
(1+\frac{1}{100})^{100} \approx 2.7048 \\
(1+\frac{1}{1000})^{1000} \approx 2.7169 \\
(1+\frac{1}{10000})^{10000} \approx 2.7181 \\
(1+\frac{1}{10000})^{100000} \approx 2.7182 \
\end{flalign*}

\par The value of e is 2.71828... The formal definition of e is therefore:

\begin{flalign}
e = \lim_{x \to \infty} (1+ \frac{1}{n})^n
\end{flalign}


\end{document}
