\documentclass{article}
\usepackage{amsmath}
\usepackage[margin=1in]{geometry}

\begin{document}

\title{Limits}
\author{Andy Chong Sam}
\date{}
\maketitle

\section{Definition}

The limit of f(x) is L as x approaches a: 

\begin{flalign}
 \lim_{x \to a} f(x) = L
\end{flalign}

\noindent The relationship described in (1) above assumes that the we get the same value L as we approach from both sides of of a. A left hand (2) or right 
hand (3) limit can be described as well:

\begin{flalign}
 \lim_{x \to a-} f(x) = L
\end{flalign}

\begin{flalign}
 \lim_{x \to a+} f(x) = L
\end{flalign}

\section{Properties of Limits}

Complex functions can be more easily evaluated using several properties. If the function is continuous at a:

\begin{flalign*}
\lim_{x \to a} f(x) = a
\end{flalign*}

\noindent For a constant k:

\begin{flalign*}
\lim_{x \to a} kf(x) = k \lim_{x \to a} f(x) 
\end{flalign*}

\noindent For sum and differences:

\begin{flalign*}
\lim_{x \to a} (f(x) + g(x)) =  \lim_{x \to a} f(x) +  \lim_{x \to a} g(x)
\end{flalign*}
\begin{flalign*}
\lim_{x \to a} (f(x) - g(x)) =  \lim_{x \to a} f(x) -  \lim_{x \to a} g(x)
\end{flalign*}

\noindent For products and quotients:

\begin{flalign*}
 \lim_{x \to a} f(x) g(x) =  \lim_{x \to a} f(x)  \lim_{x \to a} g(x)
\end{flalign*}
\begin{flalign*}
 \lim_{x \to a} \frac{f(x)}{g(x)} = \frac {\lim_{x \to a} f(x)}{\lim_{x \to a} g(x)}
\end{flalign*}

\end{document}